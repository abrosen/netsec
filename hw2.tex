\documentclass[a4paper,10pt]{article}
\usepackage[utf8]{inputenc}
\usepackage{amsmath, amssymb}
\title{HW2}
\author{Andrew Rosen}
\date{\today}
\begin{document}
\section{Question 1}

\subsection{Part a}
$$P = (C \boxplus \bar{K_{1}}) \oplus K_{0}$$ where $\bar{K_{1}}$ is the complement of $K_{1}$. 
\subsection{Part b}
Let our plaintext messages be $P_{a}, P_{b} $ and their corresponding ciphertexts be $C_{a}, C_{b}$.  We know that

$$ C_{a} = (P_{a} \oplus K_{0}) \boxplus K_{1} $$
$$ C_{b} = (P_{b} \oplus K_{0}) \boxplus K_{1} $$

We can define $K_{0}$ in terms of $C_{a}$, $P_{a}$, and $K_{1}$ by rearranging the variables.   
$$ (P_{a} \oplus K_{0}) \boxplus K_{1}  = C_{a} $$
$$  P_{a} \oplus K_{0} = C_{a} \boxplus \bar{K_{1}} $$
$$  K_{0} = (C_{a} \boxplus \bar{K_{1}}) \oplus P_{a} $$

Likewise,
$$  K_{0} = (C_{b} \boxplus \bar{K_{1}}) \oplus P_{b} $$

Which means

$$(C_{a} \boxplus \bar{K_{1}}) \oplus P_{a} = (C_{b} \boxplus \bar{K_{1}}) \oplus P_{b} $$
$$(C_{a} \boxplus \bar{K_{1}}) \oplus P_{a} \oplus P_{b} = C_{b} \boxplus \bar{K_{1}}$$
$$\left(\left(C_{a} \boxplus \bar{K_{1}}\right) \oplus P_{a} \oplus P_{b}\right) \boxplus C_{b} =  \bar{K_{1}}$$

We can then use  $\bar{K_{1}}$ to find $K_{0}$ via
$$  K_{0} = (C_{a} \boxplus \bar{K_{1}}) \oplus P_{a} $$

\end{document}
