\documentclass[a4paper,10pt]{article}
\usepackage[utf8]{inputenc}
\usepackage{amsmath, amssymb}
\usepackage{mathrsfs}
\title{HW2}
\author{Andrew Rosen}
\date{\today}
\begin{document}
\maketitle
\section{Question 1}

\subsection{Part a}
$$P = (C \boxplus \bar{K_{1}}) \oplus K_{0}$$ where $\bar{K_{1}}$ is the complement of $K_{1}$. 
\subsection{Part b}
Let our plaintext messages be $P_{a}, P_{b} $ and their corresponding ciphertexts be $C_{a}, C_{b}$.  We know that

$$ C_{a} = (P_{a} \oplus K_{0}) \boxplus K_{1} $$
$$ C_{b} = (P_{b} \oplus K_{0}) \boxplus K_{1} $$

We can define $K_{0}$ in terms of $C_{a}$, $P_{a}$, and $K_{1}$ by rearranging the variables.   
$$ (P_{a} \oplus K_{0}) \boxplus K_{1}  = C_{a} $$
$$  P_{a} \oplus K_{0} = C_{a} \boxplus \bar{K_{1}} $$
$$  K_{0} = (C_{a} \boxplus \bar{K_{1}}) \oplus P_{a} $$

Likewise,
$$  K_{0} = (C_{b} \boxplus \bar{K_{1}}) \oplus P_{b} $$

Which means

$$(C_{a} \boxplus \bar{K_{1}}) \oplus P_{a} = (C_{b} \boxplus \bar{K_{1}}) \oplus P_{b} $$
$$(C_{a} \boxplus \bar{K_{1}}) \oplus P_{a} \oplus P_{b} = C_{b} \boxplus \bar{K_{1}}$$
$$\left(\left(C_{a} \boxplus \bar{K_{1}}\right) \oplus P_{a} \oplus P_{b}\right) \boxplus C_{b} =  \bar{K_{1}}$$

We can then use  $\bar{K_{1}}$ to find $K_{0}$ via
$$  K_{0} = (C_{a} \boxplus \bar{K_{1}}) \oplus P_{a} $$
\section{Question 2}

I will solve this problem for the general case.  Let me denote the encryption of message $m$ as $E(m)$, since the key is not relevent here.  The defined encryption scheme being linear means 
$$E(m_{a}) \oplus E(m_{b})  =  E(m_{a} \oplus m_{b})$$

This also implies that a message made of all zeroes will be encrypted to a message of all zeroes.

Now, let each input $m_{i}$ and corresponding output $E(m_{i})$ be $l$ bits long.  Choose $l$ ciphertexts $E(m_{i})$ such that

$$E(m_{1}) = 1000\dots$$
$$E(m_{2}) = 0100\dots$$
$$E(m_{3}) = 0010\dots$$
$$\cdots{}$$
$$E(m_{l-1}) = \dots0010$$
$$E(m_{l}) = \dots0001$$

I denote this set $\mathscr{E}$. If I know the corresponding plaintexts $\{m_{1}, m_{2}, \dots m_{l}  \}$, I can decipher any message using the linear property of $E$. Let $E(m_{k})$ be an intercepted message. $E(m_{k}$ can be described by XORing a unique subset of  $\mathscr{E}$.  Now becuase of the linear property of $E$, I can retreive $m_{k}$ by XORing the $m$'s that correspond to the aforementioned unique subset.  

For example, let $$E(m_{k}) = E(m_{1}) \oplus E(m_{17}) \oplus E(m_{42}) $$ 
We can retrieve $m_{k}$ via

$$m_{k} = m_{1} \oplus m_{17} \oplus m_{42} $$

\end{document}
