\documentclass[a4paper,10pt]{article}
\usepackage[utf8]{inputenc}
\usepackage{amsmath, amssymb}
\usepackage{mathrsfs}
\title{HW2}
\author{Andrew Rosen}
\date{\today}
\begin{document}
\maketitle
\section{Question 1}

\subsection{Part a}
$$P = (C \boxplus \bar{K_{1}}) \oplus K_{0}$$ where $\bar{K_{1}}$ is the complement of $K_{1}$. 
\subsection{Part b}
Let our plaintext messages be $P_{a}, P_{b} $ and their corresponding ciphertexts be $C_{a}, C_{b}$.  We know that

$$ C_{n} = (P_{n} \oplus K_{0}) \boxplus K_{1} $$

We can define $K_{0}$ in terms of $C_{a}$, $P_{a}$, and $K_{1}$ by rearranging the variables.   
$$ (P_{n} \oplus K_{0}) \boxplus K_{1}  = C_{n} $$
$$  P_{a} \oplus K_{n} = C_{a} \boxplus \bar{K_{n}} $$
$$  K_{0} = (C_{n} \boxplus \bar{K_{1}}) \oplus P_{n} $$

Which means you can't solve it without solving $\bar{K_{1}}$ or ${K_{1}}$.

$$(C_{a} \boxplus \bar{K_{1}}) \oplus P_{a} = (C_{b} \boxplus \bar{K_{1}}) \oplus P_{b} $$
$$(C_{a} \boxplus \bar{K_{1}}) \oplus P_{a} \oplus P_{b} = C_{b} \boxplus \bar{K_{1}}$$
$$\left(\left(C_{a} \boxplus \bar{K_{1}}\right) \oplus P_{a} \oplus P_{b}\right) \boxplus C_{b} =  \bar{K_{1}}$$

The solution is intractably self-referential.  There is no easy solution for $K_{0}$ or $K_{1}$.  Or at least non I've found after 5 hours of working on this problem.
\pagebreak

\section{Question 2}

I will solve this problem for the general case.  Let me denote the encryption of message $m$ as $E(m)$, since the key is not relevent here.  The defined encryption scheme being linear means 
$$E(m_{a}) \oplus E(m_{b})  =  E(m_{a} \oplus m_{b})$$

This also implies that a message made of all zeroes will be encrypted to a message of all zeroes.

Now, let each input $m_{i}$ and corresponding output $E(m_{i})$ be $l$ bits long.  Choose $l$ ciphertexts $E(m_{i})$ such that

$$E(m_{1}) = 1000\dots$$
$$E(m_{2}) = 0100\dots$$
$$E(m_{3}) = 0010\dots$$
$$\cdots{}$$
$$E(m_{l-1}) = \dots0010$$
$$E(m_{l}) = \dots0001$$

I denote this set $\mathscr{E}$. If I know the corresponding plaintexts $\{m_{1}, m_{2}, \dots m_{l}  \}$, I can decipher any message using the linear property of $E$. Let $E(m_{k})$ be an intercepted message. $E(m_{k}$ can be described by XORing a unique subset of  $\mathscr{E}$.  Now becuase of the linear property of $E$, I can retreive $m_{k}$ by XORing the $m$'s that correspond to the aforementioned unique subset.  

For example, let $$E(m_{k}) = E(m_{1}) \oplus E(m_{17}) \oplus E(m_{42}) $$ 
We can retrieve $m_{k}$ via

$$m_{k} = m_{1} \oplus m_{17} \oplus m_{42} $$



\section{Question 3}
\subsection{i} Since Bob already knows $k$ and he knows the length of $v$, $(v||c)$ is effectively $(v,c)$, yielding:
$$m = \text{RC4}(v || k) \oplus c$$.
\subsection{ii}
We can detect the same input stream just by trivially looking for two messages where the $v$'s match.


\section{Question 4}
All RSA operations with a given key occur in the same mod space.  For brevity, assume that every mathematical operation below has an unwritten mod $n$ as a component. 

This question can be solved via expansion.  We have been given $B_{1}$ and $B_{2}$, as well as $C_{1}$, as well as key$\{e,n\}$.  Let us solve for $C_{2}$ such that $RSAH(C_{1},C_{2}) = RSAH(B_{1},B_{2})$.

$$RSAH(C_{1},C_{2}) = RSA(RSA(B_{1}) \oplus B_{2} ) = RSA( B_{1}^{e} \oplus B_{2} ) $$

$$RSA(RSA(C_{1}) \oplus C_{2} ) =  RSA(B_{1}^{e} \oplus B_{2} )$$

$$(C_{1}^{e} \oplus C_{2} )^{e} =  (B_{1}^{e} \oplus B_{2} )^{e}$$

$$C_{1}^{e} \oplus C_{2} =  B_{1}^{e} \oplus B_{2} $$

$$C_{2} =  C_{1}^{e} \oplus (B_{1}^{e} \oplus B_{2}) $$


Thus we can always choose a $C_{2}$ s.t. $RSAH(C_{1},C_{2}) = RSAH(B_{1},B_{2})$, and $RSAH$ does not satisfy weak collision resistance $\blacksquare$.
\pagebreak

\section{Question 5}

\subsection{Message Integrity}
In both cases Bob will see his computed $auth(x)$ won't match the sent $auth(x)$.

\subsection{Replay Attack} 
From Bob's perspective, Alice sent him 101 messages that just happen to be identical.  He won't detect Oscar unless the flood of identical messages itself registers as unusual to him. 

\subsection{Sender Authentication with cheating 3rd party}
Assume the question means that Bob has $x$ and is trying to verify who the sent the message, Alice or Oscar (the other alternative would be solving $\exists x$, which is trivial because Bob just checks to see if $x$ is there).  A message signed by the senders private key can be decrypted with (and only with, given large enough keys), the corresponding public key.  Bob can just check $auth(x)$ with Alice and Oscar's public keys to see who is telling the truth.

MACs are similarly checked, but check the secret key to generate $auth(x)$ instead.  So long as the keys Alice and Oscar are using aren't the same, it will be easy to verify who sent what.

\subsection{Authentication with Bob cheating}

If Alice sent $x$ using a digital signature, then $auth(x)$ will be generated with her private key. If Alice can show that $auth(x)$ can only be verified via her public key, then Bob is lying.

Alice won't be able to prove this in the same way using MACs, however.

\section{Question 6}

Attached is a drawing




\end{document}
